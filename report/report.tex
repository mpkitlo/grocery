\documentclass[12pt,a4paper]{article}

\usepackage[utf8]{inputenc}
\usepackage{polski}
\usepackage[parfill]{parskip}
\usepackage[top=2cm, bottom=3cm, left=2cm, right=2cm]{geometry}

\begin{document}

\begin{Large}
    Idee i informatyka, raport
	\hfill
	Zespół 3
\end{Large}
\rule{\columnwidth}{1pt}

W odpowiedzi na rosnące zapotrzebowanie na wygodne zakupy spożywcze online, nasz zespół zamierza
stworzyć aplikację \textbf{Online Grocery Shop}, której celem jest uproszczenie procesu robienia
zakupów spożywczych zarówno w sklepach stacjonarnych, jak i online. Aplikacja oferuje funkcje,
takie jak możliwość skanowania etykiet produktów i znajdowania podobnych,
tworzenie list zakupowych na podstawie wcześniejszych zakupów, rekomendacje oparte na
sztucznej inteligencji oraz system dostawy
bezpośrednio do drzwi użytkownika. Dzięki tym rozwiązaniom aplikacja umożliwia zaoszczędzenie
czasu i dostosowuje proces zakupów do indywidualnych potrzeb użytkowników.

% TODO change
\subsection*{Dla kogo jest nasz produkt?}
\begin{itemize}
    \item \textbf{Zapracowani profesjonaliści}: Osoby, które prowadzą intensywne życie zawodowe i często nie mają czasu na tradycyjne zakupy. Preferują zakupy online, ale szukają rozwiązań, które pozwolą im zaoszczędzić czas i dokonać świadomych wyborów produktowych.
    \item \textbf{Rodziny}: Osoby, które regularnie robią zakupy spożywcze i potrzebują narzędzi, które ułatwią im organizację zakupów i zaoszczędzą czas, zwłaszcza w codziennej logistyce domowej.
    \item \textbf{Seniorzy}: Osoby starsze, które mogą mieć trudności z poruszaniem się po dużych sklepach stacjonarnych, ale chętnie korzystają z zakupów online, zwłaszcza jeśli oferują one proste i łatwe w obsłudze rozwiązania.
    \item \textbf{Osoby dbające o zdrowie}: Użytkownicy, którzy preferują zdrową żywność i chcą mieć dostęp do pełnych informacji na temat jakości produktów, wartości odżywczych oraz opinii innych użytkowników.
\end{itemize}

\subsection*{Co zapewniamy?}
\begin{itemize}
    \item \textbf{Oszczędność czasu}: Dzięki funkcjom, takim jak automatyczne generowanie list
        zakupowych na podstawie wcześniejszych zakupów oraz rekomendacje najlepszych produktów
        wśród wybranej kategorii oparte na sztucznej inteligencji, użytkownicy mogą szybko i
        efektywnie realizować zakupy.
    \item \textbf{Wygodne zakupy}: Funkcja skanowania etykiet produktów pozwala na szybkie
        porównanie jakości produktów bez potrzeby przeszukiwania internetu.
    \item \textbf{Świadome zakupy}: Aplikacja zapewnia dostęp do opinii innych użytkowników oraz informacji o jakości produktów, co pozwala na podejmowanie bardziej świadomych decyzji zakupowych.
\end{itemize}

\subsection*{Czym nasze rozwiązanie wyróżnia się?}

Nasza aplikacja wyróżnia się kilkoma unikalnymi funkcjami, które sprawiają, że jest bardziej przyjazna dla użytkowników niż inne rozwiązania na rynku:
\begin{itemize}
    \item \textbf{Skanowanie etykiet}: Użytkownicy mogą zeskanować etykietę produktu, aby
        natychmiast uzyskać dostęp do recenzji innych użytkowników oraz informacji o jakości
        produktu. Funkcja ta daje użytkownikom pewność co do jakości produktów przed zakupem.
    \item \textbf{Listy zakupowe}: Sztuczna inteligencja pomaga tworzyć listy zakupowe na podstawie wcześniejszych zakupów oraz personalizowanych sugestii, co umożliwia łatwiejsze zakupy, a także przewiduje potrzeby użytkowników.
    \item \textbf{Tworzenie listy z paragonu}: Funkcja umożliwiająca generowanie list zakupowych na podstawie zeskanowanego paragonu, co pozwala zaoszczędzić czas przy ponownych zakupach.
\end{itemize}

Te funkcje wprowadzają zupełnie nowe podejście do zakupów spożywczych, które łączy wygodę, szybkość i świadomość wyborów produktowych.

\subsubsection*{Istniejąca konkurencja:}
\begin{itemize}
    \item \textbf{Allegro, Tesco, Auchan, Carrefour}: Popularne platformy e-commerce, któr
        oferują zakupy spożywcze online. Mają szeroki asortyment produktów, ale nie oferują
        specjalistycznych funkcji takich jak rekomendacje oparte na AI.
    \item \textbf{Instacart}: Usługa dostarczania produktów spożywczych z lokalnych sklepów
        działa na rynku amerykańskim. Choć oferuje szybkie dostawy, nie zawiera funkcji, które
        pomagają w tworzeniu list zakupowych na podstawie historii zakupów.
\end{itemize}

Słabą stroną rozwiązań konkurentów jest niski poziom personalizacji zakupów, co zamierzamy naprawić
wykorzystując narzędzia sztucznej inteligencji.

\subsection*{Opis sposobu realizacji rozwiązania}

Aby skutecznie zrealizować ten projekt, nasz zespół zaplanował następujący sposób realizacji
rozwiązania:


\subsubsection*{1. Implementacja głównych funkcji}

W tej fazie skupiamy się na technicznym opracowaniu aplikacji i jej kluczowych elementów:
\begin{itemize}
    \item \textbf{Frontend (Interfejs użytkownika)}: Aplikacja mobilna zostanie zaprojektowana w
        \textbf{React Native}, co pozwala na szybkie tworzenie aplikacji na systemy \textbf{iOS} i
        \textbf{Android} z jednym wspólnym kodem źródłowym.
        Aplikacja będzie miała intuicyjny interfejs z funkcjami przeglądania produktów,
        skanowania etykiet, tworzenia list zakupowych oraz zarządzania dostawą.
    \item \textbf{Backend (Serwer i bazy danych)}:
    \begin{itemize}
        \item \textbf{Node.js} z frameworkiem \textbf{Express} zostaną użyte do budowy backendu,
            który będzie obsługiwał zapytania użytkowników.
        \item \textbf{MongoDB} posłuży jako baza danych do przechowywania informacji o produktach,
            użytkownikach, historii zakupów, recenzjach oraz transakcjach. Wybór bazy NoSQL pozwala
            na szybkie przetwarzanie danych i łatwą skalowalność systemu.
    \end{itemize}
    \item \textbf{Algorytmy AI}:
    \begin{itemize}
        \item Wykorzystamy narzędzia sztucznej inteligencji, takie jak \textbf{TensorFlow} i
            \textbf{scikit-learn}, do personalizacji list zakupowych oraz rekomendacji produktów.
            Na podstawie historii zakupów i preferencji użytkownika, AI będzie proponować
            najbardziej odpowiednie produkty, co zautomatyzuje proces zakupowy i przyspieszy
            podejmowanie decyzji.
    \end{itemize}
\end{itemize}

\subsubsection*{2. Integracja z zewnętrznymi systemami}

Aplikacja wymaga integracji z szeregiem zewnętrznych systemów, aby zapewnić pełną funkcjonalność:
\begin{itemize}
    \item \textbf{Systemy do skanowania etykiet}: Zintegrujemy aplikację z technologiami,
        umożliwiającym użytkownikom skanowanie kodów kreskowych w sklepach stacjonarnych,
        aby uzyskać dostęp do recenzji, jakości produktów i danych odżywczych w czasie rzeczywistym.
    \item \textbf{Integracja z platformami e-commerce}: Aplikacja będzie łączyć się z systemami
        zewnętrznymi, np. \textbf{Allegro}, \textbf{Carrefour}, czy \textbf{Tesco}, aby
        automatycznie pobierać dane o produktach, cenach i dostępności w sklepach internetowych.
    \item \textbf{Integracja z systemami dostawców}: Aby umożliwić realizację zamówień, aplikacja
        będzie współpracować z zewnętrznymi usługami kurierskimi, takimi jak \textbf{GLS} czy
        \textbf{Poczta Polska}, które umożliwią dostarczenie produktów do drzwi użytkownika.
\end{itemize}

\subsubsection*{3. Testowanie i weryfikacja}

Wszystkie komponenty aplikacji będą poddane szczegółowym testom, aby upewnić się, że działają
zgodnie z wymaganiami:
\begin{itemize}
    \item \textbf{Testy jednostkowe i integracyjne}: Wykorzystamy narzędzia takie jak \textbf{Jest}
        i \textbf{Mocha} do testowania funkcji backendowych i frontendowych aplikacji.
    \item \textbf{Testy użyteczności}: Przeprowadzimy testy z użytkownikami, aby upewnić się,
        że interfejs jest intuicyjny, a wszystkie funkcje działają poprawnie.
    \item \textbf{Testy obciążeniowe}: Aby zapewnić skalowalność aplikacji, przeprowadzimy
        testy wydajnościowe, symulując dużą liczbę użytkowników.
\end{itemize}

\subsubsection*{4. Wdrożenie i uruchomienie aplikacji}

Po zakończeniu testów aplikacja zostanie wdrożona na platformach \textbf{Google Play} i
\textbf{Apple App Store}. Proces wdrożenia obejmuje:
\begin{itemize}
    \item \textbf{Stworzenie wersji beta}: Udostępnienie aplikacji w wersji beta wybranym
        użytkownikom, aby zebrać opinie i dokonać ostatnich poprawek.
    \item \textbf{Wdrożenie na żywo}: Publikacja aplikacji na szeroką skalę, z monitoringiem
        wydajności oraz szybkim reagowaniem na zgłaszane problemy.
\end{itemize}

\subsubsection*{5. Utrzymanie i rozwój aplikacji}

Po uruchomieniu aplikacji, będziemy kontynuować jej rozwój oraz zapewniać jej wsparcie:
\begin{itemize}
    \item \textbf{Aktualizacje}: Regularne aktualizacje aplikacji, które będą obejmować poprawki
        błędów, nowe funkcje oraz aktualizacje zgodności z systemami zewnętrznymi.
    \item \textbf{Wsparcie techniczne}: Zespół wsparcia technicznego będzie dostępny, aby
        rozwiązywać problemy użytkowników oraz zapewniać odpowiedzi na ich pytania.
    \item \textbf{Zbieranie opinii}: Aktywnie zbierać będziemy opinie użytkowników, co pozwoli na
        dalsze dostosowywanie aplikacji do ich potrzeb i oczekiwań.
\end{itemize}

% TODO
\subsection*{Szacowanie kosztów i finansowanie (biznesplan)}

% TODO
\subsection*{Badanie społeczne}

\subsection*{Jak utrzymujemy kontakt z klientami?}

Planujemy dotrzeć do potencjalnych użytkowników naszej aplikacji poprzez:
\begin{itemize}
    \item Reklamy na Facebooku, Instagramie i Google Ads
    \item Współprace z influencerami oraz blogerami
    \item Kampanie mailingowe oraz promocje w aplikacjach mobilnych
    \item Partnerstwa z sieciami sklepów
\end{itemize}

\subsection*{Podsumowanie}

Nasza aplikacja ma na celu zmianę sposobu robienia zakupów spożywczych.
W ciągu kilku lat planujemy zaimplementować wszystkie opisane funkcje.
Naszym celem jest także rozbudowa aplikacji oraz wprowadzenie nowych technologii,
które jeszcze bardziej ulepszą doświadczenie zakupowe użytkowników.

\end{document}
